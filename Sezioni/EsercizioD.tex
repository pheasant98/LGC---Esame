\pagebreak
\subsection{Esercizio D} 
Dimostrare che la trace equivalence è una congruenza per il CCS.

Prima di illustrare la dimostrazione si definisce che cosa si intende con i concetti di trace equivalence e congruenza.

Innanzitutto, per tracce di un processo P, che di seguito verrà indicata con \tr{P}, si intendono tutte le possibili sequenze di interazioni $\alpha_{1}.....\alpha_{n} \in Act$ con n >= 0 tali che esiste una sequenza di transizioni $P \overset{\alpha_{1}}\rightarrow P_{1} \overset{\alpha_{2}}\rightarrow...\overset{\alpha_{n}}\rightarrow P_{n}$, e quindi rappresenta tutte le possibili interazioni con un processo. Più formalmente \tr{P} = \{ $\alpha_{1}.....\alpha_{n} | P \overset{\alpha_{1}}\rightarrow P_{1} \overset{\alpha_{2}}\rightarrow...\overset{\alpha_{n}}\rightarrow P_{n}$ \}. Quindi due processi P e Q si dicono trace equivalent P$\sim_{t}$Q se \tr{P} = \tr{Q}.

Per congruenza si intende che, dati due processi P e Q in relazione tra loro (P \textit{R} Q), per ogni contesto C[ ], C[P] \textit{R} C[Q]. 

Perciò si dimostrerà che se P$\sim_{t}$Q $\Rightarrow \forall$C[ ] C[P] $\sim_{t}$C[Q].

\subsubsection{Dimostrazione} 

Siano P,Q e R processi CCS con P$\sim_{t}$Q, allora 

\begin{enumerate}
	\item $\alpha$.P $\sim_{t}$ $\alpha$.Q
	\item P + R $\sim_{t}$ Q + R
	\item P|R $\sim_{t}$ Q|R
	\item P\textbackslash L $\sim_{t}$ Q\textbackslash L
	\item P$\mathbf{[f]}\sim_{t}$ Q$\mathbf{[f]}$
\end{enumerate}

Si definiscono di seguito alcune terminologie che verranno usate durante la dimostrazione.

Si indica con $\varepsilon$ la sequenza vuota di interazioni. Essendo vuota, tutti i processi CCS sono in grado di eseguirla.

Sia R un processo CCS, indichiamo con $\alpha$.\tr{R} l'insieme \{$\alpha$t | t $\in$ \tr{R}\}

\paragraph{Prefisso C[ ] = $\alpha$.} \mbox{}

Nel caso del contesto prefisso si ha che \tr{$\alpha.P$} = $\alpha$.\tr{P}.
Se questo è vero allora, grazie all'ipotesi \tr{P} = \tr{Q} si ha che  $\alpha$.\tr{P} =  $\alpha$.\tr{Q} = \tr{$\alpha.Q$} e quindi vale che\\ $\alpha$.P $\sim_{t}$ $\alpha$.Q\\

$(\subseteq)$ \\

Sia  t $\in$ \tr{$\alpha.P$} $\Rightarrow$  t $\in$ $\alpha$.\tr{P}

Per induzione su |t|:
\\

\textbf{Caso Base |t| = 0}

Allora t = $\varepsilon \in$ $ \alpha$.\tr{P}
\\

\pagebreak
\textbf{Caso Induttivo |t| = n + 1}

t ha una prima interazione seguita poi dalla traccia t'. Sappiamo che $\alpha.$P $\overset{\alpha}\rightarrow $ P $\overset{t'}\rightarrow$, ovvero $\alpha$.P sa fare $\alpha$ grazie alla regola del prefisso che ne permette la transizione.

	$\dfrac{}{\alpha.P \overset{\alpha}\rightarrow P\overset{t'}\rightarrow}$ \textit{ACT} \\
	
t = $\alpha$.t' dove t' sarà una certa sequenza di interazioni con |t'| = n e per ipotesi induttiva t'$\in$ \tr{P}. Quindi vale che t = $\alpha$.t'$\in$ $\alpha$.\tr{P}.\\

$(\supseteq)$ \\

Sia  t $\in$ $\alpha$.\tr{P} $\Rightarrow$  t $\in$ \tr{$\alpha.P$}

Suddividiamo il problema in due casi:
\\

\textbf{Caso t = $\varepsilon$}

Per definizione t = $\varepsilon \in $ \tr{$\alpha.P$}.
\\
 
\textbf{Caso t $\not=$ $\varepsilon$}

Per definizione di $\alpha$.\tr{P}, t = $\alpha$.t' con t'$\in$\tr{P}. È quindi possibile effettuare la seguente sequenza di transizioni $\alpha$.P $ \overset{\alpha}\rightarrow$ P $\overset{t'}\rightarrow $. Questo dimostra che, se $\alpha$.P sa fare l'interazione t allora t = $\alpha$.t' $\in$\tr{$\alpha$.P}.

Perciò si è dimostrato che \tr{$\alpha.P$} = $\alpha$.\tr{P} e quindi con \tr{P} = \tr{Q}, $\alpha$.P $\sim_{t}$ $\alpha$.Q.
\\

\paragraph{Contesto non deterministico  C[ ] = (\hspace{0.3cm} + R)} \mbox{}

Nel caso del contesto non deterministico tra i processi P e R le \tr{P + R} = \tr{P} $\bigcup$ \tr{R}. Se questo è vero, dato che per ipotesi \tr{P} = \tr{Q} e \tr{Q + R} = \tr{Q} $\bigcup$ \tr{R}, allora \tr{P + R} = \tr{Q + R} .

Perciò si deve dimostrare che il contesto non deterministico tra i processi P e R è uguale alla unione delle traccie dei due processi. Dimostrato questo, ne consegue la veridicità di P + R $\sim_{t}$ Q + R.\\

$(\subseteq)$ \\

Sia t $\in$ \tr{P + R}  $\Rightarrow$  t $\in$ (\tr{P} $\bigcup$ \tr{R})\\
Per induzione su |t|:
\\

\textbf{Caso Base |t| = 0}

Allora t = $\varepsilon \in$(\tr{P} $\bigcup$ \tr{R})
\\

\textbf{Caso Induttivo |t| = n + 1}

t ha una prima interazione seguita poi dalla traccia t', quindi P + R $ \overset{\alpha_{1}}\rightarrow $ X' $\overset{t'}\rightarrow$ ovvero viene applicata una transizione secondo la regola della somma non deterministica arrivando in certo processo X'. Ci sono perciò due possibilità:

\begin{itemize}
	\item P+R ha effettuato una transizione usando la regola SUM-L:\\
	
	 	$\dfrac{P \overset{\alpha_{1}}\rightarrow P'}{P + R \overset{\alpha_{1}}\rightarrow P'\overset{t'}\rightarrow}$ \textit{SUM-L} \\
	 	
	 	t = $\alpha_{1}.$t' dove t' è una certa sequenza di interazioni con |t'| = n. Per ipotesi induttiva t' $\in$ \tr{P'}, quindi 
	 	$\alpha_{1}.$t' $\in \alpha_{1}.$\tr{P'} e ne consegue che t $\in$ $\alpha_{1}.$\tr{P'}. Alla luce di ciò e grazie alla regola SUM-L che permette la transizione P $\overset{\alpha_{1}}\rightarrow $ P', si può concludere che  t $\in$ \tr{P} .
	 	\\
	 	
	 \item P+R ha effettuato una transizione usando la regola SUM-R:\\
	 
	 $\dfrac{R \overset{\alpha_{1}}\rightarrow R'}{P + R \overset{\alpha_{1}}\rightarrow R'\overset{t'}\rightarrow}$ \textit{SUM-R} \\
	 
		t = $\alpha_{1}.$t' dove t' è una certa sequenza di interazioni con |t'| = n. Per ipotesi induttiva t' $\in$ \tr{R'}, quindi 
	$\alpha_{1}.$t' $\in \alpha_{1}.$\tr{R'} e ne consegue che t $\in$ $\alpha_{1}.$\tr{R'}. Alla luce di ciò e grazie alla regola SUM-R che permette la transizione R $\overset{\alpha_{1}}\rightarrow $ R', si può concludere che  t $\in$ \tr{R} .
	\\
	 	
\end{itemize}

$(\supseteq)$\\

Sia t $\in$(\tr{P} $\bigcup$ \tr{R}) $\Rightarrow $  t $\in$\tr{P + R}\\
t può essere una traccia sia di P e sia di R oppure solo uno dei due.

Suddividiamo il problema in due casi:
\\

\textbf{Caso t = $\varepsilon$}

Per definizione t = $\varepsilon \in $ \tr{P + R}.
\\

\textbf{Caso t $\not=$ $\varepsilon$}

t è una sequenza non vuota di interazioni di P.

Se t $\in$ \tr{P}, si sa che esiste la sequenza di transizione P $\overset{\alpha_{1}}\rightarrow$ P' $\overset{t'}\rightarrow$. Quindi t = $\alpha_{1}$.t' con t' $\in$ \tr{P'}. Fatta questa premessa, applicando la regola SUM-L si ottiene che P + R $\overset{\alpha_{1}}\rightarrow$ P' $\overset{t'}\rightarrow$. Si deduce che $\alpha_{1}$ $\in$ \tr{P + R}, ma allora P + R sa fare t e quindi t $\in$ \tr{P + R}.

Se t $\in$ \tr{R}, si sa che esiste la sequenza di transizione R $\overset{\alpha_{1}}\rightarrow$ R' $\overset{t'}\rightarrow$. Quindi t = $\alpha_{1}$.t' con t' $\in$ \tr{R'}. Fatta questa premessa, applicando la regola SUM-R si ottiene che P + R $\overset{\alpha_{1}}\rightarrow$ R' $\overset{t'}\rightarrow$. Si deduce che $\alpha_{1}$ $\in$ \tr{P + R}, ma allora P + R sa fare t e quindi t $\in$ \tr{P + R}.

Se t appartiene sia a P che R, qualsiasi regola venga applicata per fare la transizione vale sempre t $\in$ \tr{P + R}.\\

Perciò si è dimostrato che \tr{P + R} = \tr{P} $\bigcup$ \tr{R} e quindi con \tr{P} = \tr{Q},\\ P + R $\sim_{t}$ Q + R, come si voleva dimostrare.

\paragraph{Contesto parallelo  C[ ] = (\hspace{0.3cm} | R)} \mbox{}

Intuitivamente le traccie \tr{P|R} sono tutte le possibili combinazione tra \tr{P} e \tr{R}, cioè tutte le loro interazioni e sincronizzazioni. Se tale intuizione è vera allora dato che \tr{P} = \tr{Q}, si potrebbe sostituire P con Q nelle \tr{P|R} ed ottenere le stesse combinazioni della versione precedente. Perciò varrebbe \tr{P|R} = \tr{Q|R} e di conseguenza P|R $\sim_{t}$ Q|R.

Per dimostrare ciò, si definisce un insieme Tr(P,R) che contiene tutte e sole le sequenze di interazioni che si possono ottenere combinando \tr{P} e \tr{R} e seguendo le regole del parallelo: PAR-L, PAR-R e PAR-$\tau$.

Quindi si definisce la seguente funzione $C$ che date due traccie ne esegue la loro combinazione:
\[
	C(a,b) = 
	\begin{cases}
	max(a,b) &  $ se $ a = \varepsilon $ OR $ b = \varepsilon\\
	\alpha_{1}C(a', \alpha_{2}b')\bigcup\alpha_{2}C(\alpha_{1}a', b') & $ se $ a = \alpha_{1}a' $ AND $ b = \alpha_{2}b'\\
	\tau C(a',b') & $ se $ a = \alpha a' $ AND $ b = \Overline[1pt]{\alpha} b'\\
	\end{cases}
\]

dove Tr(P,R) = \{ $C$(a,b) | a $\in$ \tr{P}, b $\in$ \tr{R}\}.

Si procede con la dimostrazione \tr{P|R} = \tr{P,R}.\\

$(\subseteq)$ 
\\

Sia t $\in$ \tr{P|R}  $\Rightarrow$  t $\in$ \tr{P,R}.\\
Ovvero la funzione $C$ sa eseguire la traccia t.
Per induzione su |t|:\\

\textbf{Caso Base |t| = 0}

Allora t = $\varepsilon$, quindi $\varepsilon \in$ \tr{P} e $\varepsilon \in$ \tr{R} per definizione. Perciò applicando \tr{P,R}, si ha che $C(\varepsilon, \varepsilon)$ = \{ max( $\varepsilon, \varepsilon$)  \} = \{$\varepsilon$\} dove t $\in$ \{$\varepsilon$\}.\\

\textbf{Caso Induttivo |t| = n + 1}

Si ha che t = $\alpha_{1}.t'$ ovvero, t ha una prima interazione seguita poi dalla traccia t', quindi si ha P|R $ \overset{\alpha_{1}}\rightarrow $ X' $\overset{t'}\rightarrow$ con t'$\in$ \tr{X'}. Viene perciò applicata una transizione secondo la regola del parallelo arrivando in un certo processo X'. Vi sono perciò tre possibilità:

\begin{itemize}
	\item P|R ha effettuato una transizione usando la regola PAR-L:\\
	
	$\dfrac{P \overset{\alpha_{1}}\rightarrow P'}{P|R \overset{\alpha_{1}}\rightarrow P'|R\overset{t'}\rightarrow}$ \textit{PAR-L} \\
	
	t' $\in$ \tr{P'|R} con |t'| = n, per ipotesi induttiva t' $\in$ \tr{P',R} allora esistono a' $\in$ \tr{P'}, b' $\in$ \tr{R}, tale che t' $\in$ $C$(a',b'). Voglio però dimostrare che a $\in$ \tr{P}, b $\in$ \tr{R}, tale che t $\in$ $C$(a,b). Come mostrato precedentemente con l'applicazione della regola PAR-L esiste un interazione $\alpha_{1}$ tale che P $\overset{\alpha_{1}}\rightarrow$ P' con $\alpha_{1}$a' $\in$ \tr{P}. \\
	Perciò mostriamo che t $\in$ $C$(a = $\alpha_{1}$a', b = b') = \tr{P, R}. Quindi:
	\begin{itemize}
		\item Se b' = $\varepsilon$\\
		Si sa che $C$(a', $\varepsilon$) = \{a'\}, inoltre per ipotesi induttiva \\
		t' $\in$ \tr{P',R} = \{ $C$(a', $\varepsilon$) | a' $\in$ \tr{P'}, $\varepsilon$ $\in$ \tr{R}\} quindi risulta che t' = a', ma allora dato che  $\alpha_{1}$a' $\in$ \tr{P} posso concludere che t = $\alpha_{1}$t' $\in$ $C$($\alpha_{1}$a', b') e quindi t $\in$ \tr{P, R}.
		\item Se b' $\not$= $\varepsilon$\\
		Per ipotesi induttiva t' $\in$ $C$(a', b'). Si sa che dalla definizione di $C$, $C$($\alpha_{1}$a', b') può assumere il valore $\alpha_{1}C$(a', b'), ma allora t = $\alpha_{1}$.t $\in\alpha_{1}C$(a', b') e quindi t $\in$ \tr{P, R}.\\
	
		
	\end{itemize}
	
	
	\item P|R ha effettuato una transizione usando la regola PAR-R:\\
	
	$\dfrac{R \overset{\alpha_{1}}\rightarrow R'}{P|R \overset{\alpha_{1}}\rightarrow P|R'\overset{t'}\rightarrow}$ \textit{PAR-R} \\
	
		t' $\in$ \tr{P|R'} con |t'| = n, per ipotesi induttiva t' $\in$ \tr{P,R'} allora esistono a' $\in$ \tr{P}, b' $\in$ \tr{R'}, tale che t' $\in$ $C$(a',b'). Voglio però dimostrare che a $\in$ \tr{P}, b $\in$ \tr{R}, tale che t $\in$ $C$(a,b). Come mostrato precedentemente con l'applicazione della regola PAR-R esiste un interazione $\alpha_{1}$ tale che R $\overset{\alpha_{1}}\rightarrow$ R' con $\alpha_{1}$b' $\in$ \tr{R}. \\
		Perciò mostriamo che t $\in$ $C$(a = a', b = $\alpha_{1}$b') = \tr{P, R}. Quindi:
		\begin{itemize}
			\item Se a' = $\varepsilon$\\
			Si sa che $C$($\varepsilon$, b') = \{b'\}, inoltre per ipotesi induttiva \\
			t' $\in$ \tr{P,R'} = \{ $C$($\varepsilon$, b') |  $\varepsilon$ $\in$ \tr{P'}, b' $\in$ \tr{R}\} quindi risulta che t' = b', ma allora dato che  $\alpha_{1}$b' $\in$ \tr{R} posso concludere che t = $\alpha_{1}$t' $\in$ $C$(a', $\alpha_{1}$b') e quindi t $\in$ \tr{P, R}.
			\item  Se a' $\not$= $\varepsilon$\\
			Per ipotesi induttiva t' $\in$ $C$(a', b'). Si sa che dalla definizione di $C$, $C$(a', $\alpha_{1}$b') può assumere il valore $\alpha_{1}C$(a', b'), ma allora t = $\alpha_{1}$.t $\in\alpha_{1}C$(a', b') e quindi t $\in$ \tr{P, R}.\\
			
			
		\end{itemize}
	\item P|R ha effettuato una transizione usando la regola PAR-$\tau$:\\
	
	$\dfrac{P \overset{\alpha_{1}}\rightarrow P' \hspace{1cm} R \overset{\Overline[0.1pt]{\alpha_{1}}}\rightarrow R'}{P|R \overset{\tau_{1}}\rightarrow P'|R'\overset{t'}\rightarrow}$ \textit{PAR-$\tau$}
	
	t' $\in$ \tr{P'|R'} con |t'| = n, per ipotesi induttiva t' $\in$ \tr{P', R'} allora esistono a' $\in$ \tr{P'}, b' $\in$ \tr{R'}, tale che t' $\in$ $C$(a',b'). Voglio però dimostrare che a $\in$ \tr{P}, b $\in$ \tr{R}, tale che t $\in$ $C$(a,b). Come mostrato precedentemente con l'applicazione della regola PAR-$\tau$ esistono le interazioni $\alpha_{1}$ e $\Overline[0.1pt]{\alpha_{1}}$ tale che P $\overset{\alpha_{1}}\rightarrow$ P'  e R $\overset{\Overline[0.1pt]\alpha_{1}}\rightarrow$ R' con $\alpha_{1}$a' $\in$ \tr{P} e $\Overline[0.1pt]\alpha_{1}$b' $\in$ \tr{R}. \\
	Perciò mostriamo che t $\in$ $C$(a = $\alpha_{1}$a', b = $\Overline[0.1pt]\alpha_{1}$b') = \tr{P, R}. \\
	Quindi dato che si sincronizzano nessuna delle due traccie possono essere vuote, e per ipotesi induttiva t' $\in$ $C$(a', b'), sapendo che dalla definizione di $C$, $C$($\alpha_{1}$a', $\Overline[0.1pt]\alpha_{1}$b') può assumere il valore $\tau C$(a', b'), allora t = $\tau$t' $\in$ $\tau C$(a', b') e quindi t $\in$ \tr{P, R}. \\
	

\end{itemize}

$(\supseteq)$ 
\\

Sia t $\in$ \tr{P,R}  $\Rightarrow$  t $\in$ (P|R)\\
Per induzione su |t|:\\

\textbf{Caso Base |t| = 0}

t = $\varepsilon$, allora $\varepsilon \in$ \tr{P | R} per definizione.\\


\textbf{Caso Induttivo |t| = n + 1}


t ha una prima interazione $\alpha_{1}$ seguita poi dalla traccia t', ovvero t = $\alpha_{1}$t'. Si ricorda inoltre che esiste una a $\in$ \tr{P}, b $\in$ \tr{R}, tale che t $\in$ $C$(a,b). Ci sono perciò tre possibili $\alpha_{1}$:

\begin{itemize}
	\item $\alpha_{1}$ $\in$ \tr{P}, si ha quindi una sequenza di transizione P $\overset{\alpha_{1}}\rightarrow$ P' $\overset{a'}\rightarrow$ con a = $\alpha_{1}$a' con a'$\in$ \tr{P'}. Posso perciò applicare PAR-L:\\
	
	$\dfrac{P \overset{\alpha_{1}}\rightarrow P'}{P|R \overset{\alpha_{1}}\rightarrow P'|R\overset{a'}\rightarrow}$ \textit{PAR-L} \\
	
	Dall'applicazione della regola PAR-L si ottiene che P|R $\overset{\alpha_{1}}\rightarrow $ P'|R $\overset{a'}\rightarrow$. Si deduce che $\alpha_{1}$ $\in$ \tr{P|R}, |t'| = n  per ipotesi induttiva t' $\in$ \tr{P'|R}, allora (P|R) sa fare t quindi t = $\alpha_{1}$t' $\in$ \tr{P|R}.  \\
	
	\item $\alpha_{1}$ $\in$ \tr{R}, si ha quindi una sequenza di transizione R $\overset{\alpha_{1}}\rightarrow$ R' $\overset{b'}\rightarrow$ con b = $\alpha_{1}$b' con b'$\in$ \tr{R'}. Posso perciò applicare PAR-R:\\
	
	$\dfrac{R \overset{\alpha_{1}}\rightarrow R'}{P|R \overset{\alpha_{1}}\rightarrow P|R'\overset{b'}\rightarrow}$ \textit{PAR-R} \\
	
	Dall'applicazione della regola PAR-R si ottiene che P|R $\overset{\alpha_{1}}\rightarrow $ P|R' $\overset{b'}\rightarrow$. Si deduce che $\alpha_{1}$ $\in$ \tr{P|R}, |t'| = n per ipotesi induttiva t' $\in$ \tr{P|R'}, allora (P|R) sa fare t quindi t = $\alpha_{1}$t' $\in$ \tr{P|R}.\\
	
	\item $\alpha_{1}$ $\in$ \tr{P} AND $\alpha_{1}$ $\in$ \tr{R}, si ha quindi una sequenza di transizione P $\overset{\alpha_{1}}\rightarrow$ P' $\overset{a'}\rightarrow$ con a = $\alpha_{1}$.a' con a'$\in$ \tr{P'}, e un altra sequenza di transizioni R $\overset{\Overline[0.1pt]\alpha_{1}}\rightarrow$ R' $\overset{b'}\rightarrow$ con b = $ \Overline[0.1pt]\alpha_{1}.b'$ con b'$\in$ \tr{R'}. Posso perciò applicare PAR-$\tau$:\\
	
	$\dfrac{P \overset{\alpha_{1}}\rightarrow P' \hspace{1cm} R \overset{\Overline[0.1pt]{\alpha_{1}}}\rightarrow R'}{P|R \overset{\tau_{1}}\rightarrow P'|R'\overset{t'}\rightarrow}$ \textit{PAR-$\tau$}\\
	
	Dall'applicazione della regola PAR-$\tau$ si ottiene che P|R $\overset{\tau_{1}}\rightarrow $ P'|R' $\overset{t'}\rightarrow$. Si deduce che $\tau_{1}$ $\in$ \tr{P|R}, |t'| = n per ipotesi induttiva t' $\in$ \tr{P'|R'}, allora (P|R) sa fare t quindi t = $\tau_{1}$t' $\in$ \tr{P|R}.\\

	
\end{itemize}

Quindi sia \tr{P} = \tr{Q} e \tr{P,R} = \{ $C$(a,b) | a $\in$ \tr{P}, b $\in$ \tr{R}\} = \tr{P | R}, allora posso sostituire le traccie di P con quelle di Q per ottenere \{ $C$(a,b) | a $\in$ \tr{Q}, b $\in$ \tr{R}\} = \tr{P,R} = \tr{Q | R}, dimostrando che P|R $\sim_{t}$ Q|R.

\paragraph{Contesto restrizione  C[ ] = \textbackslash L} \mbox{}

Il caso del contesto restrizione L sul processo P ha la seguente uguaglianza:\\ \tr{P\textbackslash L} = \tr{P}\textbackslash\{$\alpha_{1}$...$\alpha_{n}$|$\alpha_{i}\in$L\} cioè le traccie che stanno in \tr{P} non ci sono nell'insieme di restrizione L. Se questo è vero, dato che per ipotesi \tr{P} = \tr{Q} e quindi \tr{Q}\textbackslash \{$\alpha_{1}$...$\alpha_{n}$|$\alpha_{i}\in$L\} = \tr{Q\textbackslash L}, allora \tr{P\textbackslash L} = \tr{Q\textbackslash L}. 

Perciò si deve dimostrare che il contesto restrizione L sul processo P è uguale a \\
\tr{P}\textbackslash \{$\alpha_{1}$...$\alpha_{n}$|$\alpha_{i}\in$L\}. Dimostrato questo ne consegue la veridicità di P\textbackslash L $\sim_{t}$ Q\textbackslash L.\\

$(\subseteq)$ \\

Sia t $\in$\tr{P\textbackslash L}  $\Rightarrow$  t $\in$\tr{P}\textbackslash\{$\alpha_{1}$...$\alpha_{n}$|$\alpha_{i}\in$L\}\\
Per induzione su |t|:
\\

\textbf{Caso Base |t| = 0}

Allora t = $\varepsilon \in$ \tr{P}\textbackslash\{$\alpha_{1}$...$\alpha_{n}$|$\alpha_{i}\in$L\} per definizione.
\\

\textbf{Caso Induttivo |t| = n + 1}

t ha una prima interazione seguita poi dalla traccia t', quindi P\textbackslash L $ \overset{\alpha_{1}}\rightarrow $ X' $\overset{t'}\rightarrow$ ovvero viene applicata una transizione secondo la regola della restrizione arrivando in un certo processo X', quindi:
	
	$\dfrac{P \overset{\alpha_{1}}\rightarrow P'}{P\textbackslash L \overset{\alpha_{1}}\rightarrow P'\textbackslash L\overset{t'}\rightarrow}$ \textit{RES} se $\alpha_{1}, \Overline[1pt]{\alpha_{1}} \not\in$ L\\
	
	t = $\alpha_{1}.t'$ con |t'| = n, per ipotesi induttiva t' $\in$ \tr{P'}\textbackslash \{$\alpha_{1}$...$\alpha_{n}$|$\alpha_{i}\in$L\}. \\
	Dato che $\alpha_{1}, \Overline[1pt]{\alpha_{1}} \not\in$ L quindi t = $\alpha$.t' $\not \in$ \{$\alpha_{1}$...$\alpha_{n}$|$\alpha_{i}\in$L\} allora vale che t = $\alpha_{1}.t'$ $\in$ \tr{P}\textbackslash \{$\alpha_{1}$...$\alpha_{n}$|$\alpha_{i}\in$L\}.\\


$(\supseteq)$\\

Sia  t $\in$ \tr{P}\textbackslash\{$\alpha_{1}$...$\alpha_{n}$|$\alpha_{i}\in$L\} $\Rightarrow$ t $\in$ \tr{P\textbackslash L}.   \\
Per induzione su |t|:
\\

\textbf{Caso Base |t| = 0}

Allora t = $\varepsilon \in$ \tr{P\textbackslash L} per definizione.
\\

\textbf{Caso Induttivo |t| = n + 1}

t ha una prima interazione seguita poi dalla traccia t', cioè t = $\alpha_{1}$.t'. Si ha quindi una transizione P $\overset{\alpha_{1}}\rightarrow$ P', ciò è permesso dalla regola della restrizione, quindi:

$\dfrac{P \overset{\alpha_{1}}\rightarrow P'}{P\textbackslash L \overset{\alpha_{1}}\rightarrow P'\textbackslash L\overset{t'}\rightarrow}$ \textit{RES} se $\alpha_{1}, \Overline[1pt]{\alpha_{1}} \not\in$ \{$\alpha_{1}$...$\alpha_{n}$|$\alpha_{i}\in$L\}\\


Dall'applicazione della regola RES si ottiene che P\textbackslash L $\overset{\alpha_{1}}\rightarrow$ P'\textbackslash L$\overset{t'}\rightarrow$, posso dedurre che $\alpha_{1} \in$ \tr{P\textbackslash L}. |t'| = n per ipotesi induttiva t'$\in$ \tr{P'$\textbackslash$L} allora \tr{P\textbackslash L} sa fare t, quindi t = $\alpha_{1}$.t' $\in$ \tr{P$\textbackslash$L}.\\

Dato che \tr{P\textbackslash L} = \tr{P}\textbackslash \{$\alpha_{1}$...$\alpha_{n}$|$\alpha_{i}\in$L\} con \tr{P} = \tr{Q}  allora si è dimostrato che  P\textbackslash L $\sim_{t}$ Q\textbackslash L.

\paragraph{Contesto relabelling  C[ ] = $\mathbf{[f]}$ } \mbox{}

Nel caso del contesto relabelling sul processo P si ha che:	\\	
Data la funzione $f: Act \rightarrow Act$, le traccie di P$\mathbf{[f]}$ sono:\\
f($\epsilon$) = $\epsilon$\\
f($\alpha.t$) =  f($\alpha$).f(t)\\
Quindi voglio dimostrare che \tr{P$\mathbf{[f]}$} = \{f(t)| t $\in$ \tr{P}\}. Se questo è vero, dato che \tr{P} = \tr{Q}  si può sostituire \tr{P} con \tr{Q} scrivendo \{f(t)| t $\in$ \tr{Q}\} e grazie alla uguaglianza scritta precedentemente, allora \tr{P$\mathbf{[f]}$} = \tr{Q$\mathbf{[f]}$}. \\

$(\subseteq)$ \\

Sia t$\in$ \tr{P$\mathbf{[f]}$} $\Rightarrow$ t $\in$ \{f(t)| t $\in$ \tr{P}\}. \\
Per induzione su |t|:
\\

\textbf{Caso Base |t| = 0}

Allora t = $\varepsilon \in$ \{f($\varepsilon$)| $\varepsilon$ $\in$ \tr{P}\} per definizione.\\


\textbf{Caso Induttivo |t| = n + 1}

t ha una prima interazione seguita poi dalla traccia t', quindi P$\mathbf{[f]}  \overset{\alpha_{1}}\rightarrow $ X' $\overset{t'}\rightarrow$ ovvero viene applicata una transizione secondo la regola del relabelling arrivando in un certo processo X', quindi:

$\dfrac{P \overset{\alpha_{1}}\rightarrow P'}{P \mathbf{[f]} \overset{f(\alpha_{1})}\rightarrow P'\mathbf{[f]}\overset{f(t')}\rightarrow}$ \textit{REL} 

t = $f(\alpha_{1}).f(t)'$ dove t' è una certa sequenza di interazioni con |t'| = n. Per ipotesi induttiva si ha che f(t') $\in$ \{f(t')| t' $\in$ \tr{P'}\}, inoltre per l'applicazione della regola REL che permette la transizione P $\overset{\alpha_{1}}\rightarrow$ P' allora f($\alpha$).f(t') $\in$ \{f(t)| t $\in$ \tr{P}\}.\\

$(\supseteq)$ \\

Sia t$\in$\{f(t)| t $\in$ \tr{P}\}  $\Rightarrow$ t$\in$\tr{P$\mathbf{[f]}$}. \\
Per induzione su |t|:\\

\textbf{Caso Base |t| = 0}

Allora t = $\varepsilon \in$ \tr{P$\mathbf{[f]}$} per definizione.\\

\textbf{Caso Induttivo |t| = n + 1}

t ha una prima interazione seguita poi dalla traccia t', cioè t = $\alpha_{1}$.t'. Si ha quindi una transizione P $\overset{\alpha_{1}}\rightarrow$ P', ciò è permesso dalla regola del relabelling, quindi:

$\dfrac{P \overset{\alpha_{1}}\rightarrow P'}{P \mathbf{[f]} \overset{f(\alpha_{1})}\rightarrow P'\mathbf{[f]}\overset{f(t')}\rightarrow}$ \textit{REL} 

Dall'applicazione della regola REL si ottiene che P $\mathbf{[f]} \overset{f(\alpha_{1})}\rightarrow $P'$\mathbf{[f]}\overset{f(t')}\rightarrow$. Si deduce che P$\mathbf{[f]}$ sa fare l'interazione $\alpha_{1}$, inoltre |t'| = n, per ipotesi induttiva t'$\in$ P'$\mathbf{[f]}$ allora t = $\alpha_{1}$.t' $\in$ \tr{P$\mathbf{[f]}$}.\\

Quindi dato che \tr{P$\mathbf{[f]}$} = \{f(t)| t $\in$ \tr{P}\} con \tr{P} = \tr{Q} allora si è dimostrato che  P$\mathbf{[f]}\sim_{t}$ Q$\mathbf{[f]}$.\\

\subsubsection{Conclusione}
Si è dimostrato con i vari casi della dimostrazione precedente, che per ogni possibile contesto che può essere usato, la trace equivalence risulta essere una congruenza per il CCS. 
\pagebreak