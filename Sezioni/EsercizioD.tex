\pagebreak
\subsection{Esercizio D} 
Dimostrare che la trace equivalence è una congruenza per il CCS.

Prima di illustrare la dimostrazione si definisce che cosa si intende con i concetti di trace equivalence e congruenza.

Innanzitutto per traces di un processo P che di seguito verrà indicata con \tr{P} si intende, le sequenze di interazioni $\alpha_{1}.....\alpha_{n} \in Act$ con n>= 0 tale che esiste una sequenza di transizioni $P \overset{\alpha_{1}}\rightarrow P_{1} \overset{\alpha_{2}}\rightarrow...\overset{\alpha_{n}}\rightarrow P_{n}$, e quindi rappresentata tutte le possibili interazioni con un processo. Più formalmente \tr{P} = \{ $\alpha_{1}.....\alpha_{n} | P \overset{\alpha_{1}}\rightarrow P_{1} \overset{\alpha_{2}}\rightarrow...\overset{\alpha_{n}}\rightarrow P_{n}$ \}. Quindi due processi P e Q si dicono trace equivalence P$\sim_{t}$Q se \tr{P} = \tr{Q}.
\\

Per congruenza si intende, dati due processi P e Q in relazione tra loro (P \textit{R} Q), allora per ogni contesto C[ ], C[P] \textit{R} C[Q]. 

Perciò si dimostrerà che se P$\sim_{t}$Q $\Rightarrow \forall$C[ ] C[P] $\sim_{t}$C[Q].

\subsubsection{Dimostrazione} 

Siano P,Q e R processi CCS con P $\sim_{t}$ Q, allora 

\begin{enumerate}
	\item $\alpha$.P $\sim_{t}$ $\alpha$.Q
	\item P + R $\sim_{t}$ Q + R
	\item P|R $\sim_{t}$ Q|R
	\item P\textbackslash L $\sim_{t}$ Q\textbackslash L
	\item P $\mathbf{[f]}\sim_{t}$ Q$\mathbf{[f]}$
\end{enumerate}

\paragraph{Prefisso C[ ] = $\alpha$.} \mbox{}

Si ha che \tr{C[P]} = \tr{$\alpha$.P} = $\alpha$.\tr{P}, dato che con il contesto C[ ] si è aggiunto l'interazione $\alpha$ alle \tr{P}. Per ipotesi induttiva \tr{P} = \tr{Q}, inoltre aggiungendo l'interazione $\alpha$ sia a \tr{P} e sia a \tr{Q}, si ha che $\alpha$.\tr{P} = $\alpha$.\tr{Q} = \tr{$\alpha$.Q}.

Perciò vale $\alpha$.P $\sim_{t}$ $\alpha$.Q.

\paragraph{Contesto non deterministico  C[ ] = (\hspace{0.3cm} + R)} \mbox{}

Nel caso del contesto non deterministico tra i processi P e R le \tr{P + R} = \tr{P} $\bigcup$ \tr{R}. Se questo è vero, dato che per ipotesi induttiva \tr{P} = \tr{Q} e \tr{Q + R} = \tr{Q} $\bigcup$ \tr{R}, allora \tr{P + R} = \tr{Q + R} 

Perciò si deve dimostrare che il contesto non deterministico tra i processi P e R è uguale alla unione delle traccie dei due processi. Dimostrato questo ne consegue la veridicità di P + R $\sim_{t}$ Q + R.

$(\subseteq)$ \\

Sia t $\in$\tr{P + R}  $\Rightarrow$  t $\in$(\tr{P} $\bigcup$ \tr{R})
Per induzione su |t|:
\\

\textbf{Caso Base |t| = 0}
\\
Allora t = $\varepsilon \in$(\tr{P} $\bigcup$ \tr{R})
\\

\textbf{Caso Induttivo |t| = n + 1}

t ha una prima interazione seguita poi dalla traccia t', quindi P + R $ \overset{\alpha_{1}}\rightarrow $ X' $\overset{t'}\rightarrow$ ovvero viene applicata una transizione secondo la regola della somma non deterministica arrivando in certo stato X'; ci sono perciò due possibilità:

\begin{itemize}
	\item P+R ha effettuato una transizione usando la regola SUM-L:\\
	
	 	$\dfrac{P \overset{\alpha_{1}}\rightarrow P'}{P + R \overset{\alpha_{1}}\rightarrow P'\overset{t'}\rightarrow}$ \textit{SUM-L} \\
	 	
	 	t = $\alpha_{1}.t'$, per ipotesi induttiva t' $\in \tr{P'}$, quindi 
	 	$\alpha_{1}.t' \in \tr{P}$ allora $\alpha_{1}.t' \in \tr{P} \bigcup \tr{R}$
	 	\\
	 	
	 \item P+R ha effettuato una transizione usando la regola SUM-R:\\
	 
	 $\dfrac{R \overset{\alpha_{1}}\rightarrow R'}{P + R \overset{\alpha_{1}}\rightarrow R'\overset{t'}\rightarrow}$ \textit{SUM-R} \\
	 
	 t = $\alpha_{1}.t'$, per ipotesi induttiva t' $\in \tr{R'}$, quindi 
	 $\alpha_{1}.t' \in \tr{R}$ allora $\alpha_{1}.t' \in \tr{R} \bigcup \tr{P}$.
	 	\\
	 	
\end{itemize}

$(\supseteq)$\\

Sia t $\in$(\tr{P} $\bigcup$ \tr{R}) $\Rightarrow $  t $\in$\tr{P + R}
Perciò t è una traccia sia di P e di R oppure solo uno dei due, quindi:
\\

Se t $\in$ \tr{P}, P + R può scegliere di fare una transizione attraverso la regola SUM-L, e allora vale che t $\in$ \tr{P + R}.

Se t $\in$ \tr{R}, P + R può scegliere di fare una transizione attraverso la regola SUM-R, e allora vale che t $\in$ \tr{P + R}.

Se t appartiene sia a P che R, qualsiasi regola venga applicata per fare la transizione vale sempre t $\in$ \tr{P + R}.

Perciò si è dimostrato che \tr{P + R} = \tr{P} $\bigcup$ \tr{R} e quindi con \tr{P} = \tr{Q},\\ P + R $\sim_{t}$ Q + R come si voleva dimostrare.
