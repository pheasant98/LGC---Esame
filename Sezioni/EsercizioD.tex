\pagebreak
\subsection{Esercizio D} 
Dimostrare che la trace equivalence è una congruenza per il CCS.

Prima di illustrare la dimostrazione si definisce che cosa si intende con i concetti di trace equivalence e congruenza.

Innanzitutto per tracce di un processo P che di seguito verrà indicata con \tr{P}, si intendono tutte le possibili sequenze di interazioni $\alpha_{1}.....\alpha_{n} \in Act$ con n>= 0 tale che esiste una sequenza di transizioni $P \overset{\alpha_{1}}\rightarrow P_{1} \overset{\alpha_{2}}\rightarrow...\overset{\alpha_{n}}\rightarrow P_{n}$, e quindi rappresentata tutte le possibili interazioni con un processo. Più formalmente \tr{P} = \{ $\alpha_{1}.....\alpha_{n} | P \overset{\alpha_{1}}\rightarrow P_{1} \overset{\alpha_{2}}\rightarrow...\overset{\alpha_{n}}\rightarrow P_{n}$ \}. Quindi due processi P e Q si dicono trace equivalence P$\sim_{t}$Q se \tr{P} = \tr{Q}.

Per congruenza si intende, dati due processi P e Q in relazione tra loro (P \textit{R} Q), allora per ogni contesto C[ ], C[P] \textit{R} C[Q]. 

Perciò si dimostrerà che se P$\sim_{t}$Q $\Rightarrow \forall$C[ ] C[P] $\sim_{t}$C[Q].

\subsubsection{Dimostrazione} 

Siano P,Q e R processi CCS con P $\sim_{t}$ Q, allora 

\begin{enumerate}
	\item $\alpha$.P $\sim_{t}$ $\alpha$.Q
	\item P + R $\sim_{t}$ Q + R
	\item P|R $\sim_{t}$ Q|R
	\item P\textbackslash L $\sim_{t}$ Q\textbackslash L
	\item P$\mathbf{[f]}\sim_{t}$ Q$\mathbf{[f]}$
\end{enumerate}

Si definiscono di seguito alcune terminologie che verranno usate durante la dimostrazione.

Si indica con $\varepsilon$ la sequenza vuota di interazioni. Essendo vuoto tutti i processi CCS sono in grado di eseguirla.

Sia R un processo CCS, indichiamo con $\alpha$.\tr{R} l'insieme \{$\alpha$t | t $\in$ \tr{R}\}

\paragraph{Prefisso C[ ] = $\alpha$.} \mbox{}

Nel caso del contesto prefisso si ha che \tr{$\alpha.P$} = $\alpha$.\tr{P}.
Se questo è vero allora, grazie all'ipotesi \tr{P} = \tr{Q} si ha che  $\alpha$.\tr{P} =  $\alpha$.\tr{Q} = \tr{$\alpha.Q$} e quindi vale che $\alpha$.P $\sim_{t}$ $\alpha$.Q\\

$(\subseteq)$ \\

Sia  t $\in$ \tr{$\alpha.P$} $\Rightarrow$  t $\in$ $\alpha$.\tr{P}

Per induzione su |t|:
\\

\textbf{Caso Base |t| = 0}

Allora t = $\varepsilon \in$ $ \alpha$.\tr{P}
\\

\textbf{Caso Induttivo |t| = n + 1}

t ha una prima interazione seguita poi dalla traccia t'. Sappiamo che $\alpha.P\overset{\alpha}\rightarrow $ P $\overset{t'}\rightarrow$, ovvero $\alpha$.P sa fare $\alpha$ grazie alla regola del prefisso che ne permette la transizione.

	$\dfrac{}{\alpha.P \overset{\alpha}\rightarrow P\overset{t'}\rightarrow}$ \textit{ACT} \\
	
t = $\alpha$.t' dove t' sarà una certa sequenza di interazioni dove per ipotesi induttiva t'$\in$ \tr{P}. E quindi vale che t = $\alpha$.t'$\in$ $\alpha$.\tr{P}.\\

$(\supseteq)$ \\

Sia  t $\in$ $\alpha$.\tr{P} $\Rightarrow$  t $\in$ \tr{$\alpha.P$}

Suddividiamo il problema in due casi:
\\

\textbf{Caso t = $\varepsilon$}

Per definizione t = $\varepsilon \in $ \tr{$\alpha.P$}.
\\
 
\textbf{Caso t $\not=$ $\varepsilon$}

Per definizione di $\alpha$.\tr{P}, t = $\alpha$.t' con t'$\in$\tr{P}. È quindi è possibili effettuare la seguente sequenza di transizioni $\alpha.P \overset{\alpha}\rightarrow P\overset{t'}\rightarrow $. Questo dimostra che $\alpha.P$ sa fare l'interazione t, allora t = $\alpha$.t' $\in$\tr{$\alpha$.P}.

Perciò si è dimostrato che \tr{$\alpha.P$} = $\alpha$.\tr{P} e quindi con \tr{P} = \tr{Q}, $\alpha$.P $\sim_{t}$ $\alpha$.Q come si voleva dimostrare.
\\

\paragraph{Contesto non deterministico  C[ ] = (\hspace{0.3cm} + R)} \mbox{}

Nel caso del contesto non deterministico tra i processi P e R le \tr{P + R} = \tr{P} $\bigcup$ \tr{R}. Se questo è vero, dato che per ipotesi \tr{P} = \tr{Q} e \tr{Q + R} = \tr{Q} $\bigcup$ \tr{R}, allora \tr{P + R} = \tr{Q + R} .

Perciò si deve dimostrare che il contesto non deterministico tra i processi P e R è uguale alla unione delle traccie dei due processi. Dimostrato questo ne consegue la veridicità di P + R $\sim_{t}$ Q + R.\\

$(\subseteq)$ \\

Sia t $\in$ \tr{P + R}  $\Rightarrow$  t $\in$ (\tr{P} $\bigcup$ \tr{R})\\
Per induzione su |t|:
\\

\textbf{Caso Base |t| = 0}

Allora t = $\varepsilon \in$(\tr{P} $\bigcup$ \tr{R})
\\

\textbf{Caso Induttivo |t| = n + 1}

t ha una prima interazione seguita poi dalla traccia t', quindi P + R $ \overset{\alpha_{1}}\rightarrow $ X' $\overset{t'}\rightarrow$ ovvero viene applicata una transizione secondo la regola della somma non deterministica arrivando in certo processo X'. Ci sono perciò due possibilità:

\begin{itemize}
	\item P+R ha effettuato una transizione usando la regola SUM-L:\\
	
	 	$\dfrac{P \overset{\alpha_{1}}\rightarrow P'}{P + R \overset{\alpha_{1}}\rightarrow P'\overset{t'}\rightarrow}$ \textit{SUM-L} \\
	 	
	 	t = $\alpha_{1}.$t' dove t' è una certa sequenza di interazioni con |t'| = n. Per ipotesi induttiva t' $\in$ \tr{P'}, quindi 
	 	$\alpha_{1}.$t' $\in \alpha_{1}.$\tr{P'} che ne consegue che t $\in$ $\alpha_{1}.$\tr{P'}. Alla luce di ciò e grazie alla regola SUM-L che permette la transizione P $\overset{\alpha_{1}}\rightarrow $P', si può concludere che  t $\in$ \tr{P} .
	 	\\
	 	
	 \item P+R ha effettuato una transizione usando la regola SUM-R:\\
	 
	 $\dfrac{R \overset{\alpha_{1}}\rightarrow R'}{P + R \overset{\alpha_{1}}\rightarrow R'\overset{t'}\rightarrow}$ \textit{SUM-R} \\
	 
		t = $\alpha_{1}.$t' dove t' è una certa sequenza di interazioni con |t'| = n. Per ipotesi induttiva t' $\in$ \tr{R'}, quindi 
	$\alpha_{1}.$t' $\in \alpha_{1}.$\tr{R'} che ne consegue che t $\in$ $\alpha_{1}.$\tr{R'}. Alla luce di ciò e grazie alla regola SUM-R che permette la transizione R $\overset{\alpha_{1}}\rightarrow $R', si può concludere che  t $\in$ \tr{R} .
	\\
	 	
\end{itemize}

$(\supseteq)$\\

Sia t $\in$(\tr{P} $\bigcup$ \tr{R}) $\Rightarrow $  t $\in$\tr{P + R}\\
t può essere una traccia sia di P e sia di R oppure solo uno dei due.

Suddividiamo il problema in due casi:
\\

\textbf{Caso t = $\varepsilon$}

Per definizione t = $\varepsilon \in $ \tr{P + R}.
\\

\textbf{Caso t $\not=$ $\varepsilon$}

t è una sequenza non vuota di interazioni di P.

Se t $\in$ \tr{P}, si sa che esiste la sequenza di transizione P $\overset{\alpha_{1}}\rightarrow$ P' $\overset{t'}\rightarrow$. Quindi t = $\alpha_{1}$.t' con t' $\in$ \tr{P'}.Fatta questa premessa, applicando la regola SUM-L si ottiene che $P + R \overset{\alpha_{1}}\rightarrow P'\overset{t'}\rightarrow$, allora t $\in$ \tr{P + R}.

Se t $\in$ \tr{R}, si sa che esiste la sequenza di transizione R $\overset{\alpha_{1}}\rightarrow$ R' $\overset{t'}\rightarrow$. Quindi t = $\alpha_{1}$.t' con t' $\in$ \tr{R'}. Fatta questa premessa, applicando la regola SUM-R si ottiene che $P + R \overset{\alpha_{1}}\rightarrow R'\overset{t'}\rightarrow$, allora t $\in$ \tr{P + R}

Se t appartiene sia a P che R, qualsiasi regola venga applicata per fare la transizione vale sempre t $\in$ \tr{P + R}.\\

Perciò si è dimostrato che \tr{P + R} = \tr{P} $\bigcup$ \tr{R} e quindi con \tr{P} = \tr{Q},\\ P + R $\sim_{t}$ Q + R come si voleva dimostrare.

\paragraph{Contesto parallelo  C[ ] = (\hspace{0.3cm} | R)} \mbox{}

Intuitivamente le traccie \tr{P|R} sono tutte le possibili combinazione tra \tr{P} e \tr{R}, cioè quindi tutte le loro interazioni e sincronizzazioni. Se tale intuizione è vera allora dato che \tr{P} = \tr{Q}, si potrebbe sostituire P con Q nelle \tr{P|R} ed ottenere le stesse combinazioni della versione precedente, quindi varrebbe che \tr{P|R} = \tr{Q|R} e di conseguenza P|R $\sim_{t}$ Q|R.

Per dimostrare ciò, si deve prima dimostrare il seguente lemma che sarà utilizzato nella dimostrazione:\\
Siano A e B due processi CSS, se \tr{A} = \tr{B} $\Rightarrow \forall\alpha$ \tr{A'}$\subseteq$\tr{$\displaystyle\sum_{}^{} \{B'|B \overset{\alpha}\rightarrow B'\}$} \\con A $ \overset{\alpha}\rightarrow $ A'\\
Cioè se i processi A e B hanno le stesse traccie allora le traccie del sotto processo di A, A' sono incluse nell'insieme delle traccie relative ai sotto processi B', raggiunti con una transizione $\alpha$ dal processo B.\\

Si dimostra di seguito tale lemma:

\tr{$\displaystyle\sum_{}^{} \{B'|B \overset{\alpha}\rightarrow B'\}$} = \{t'|t = $\alpha$.t'$\in$\tr{B}\}, dato che \tr{A} = \tr{B} \\
allora posso sostituire le \tr{B} con \tr{A} quindi, \{t'|t = $\alpha$.t'$\in$\tr{A}\}.\\
Perciò \tr{A'}$\subseteq$\{t'|t = $\alpha$.t'$\in$\tr{A}\} = \tr{$\displaystyle\sum_{}^{} \{B'|B \overset{\alpha}\rightarrow B'\}$} in accordo con quanto scritto precedentemente.

Si procede con la dimostrazione \tr{P|R} = \tr{Q|R}:\\

$(\subseteq)$ 
\\

Sia t $\in$\tr{P|R}  $\Rightarrow$  t $\in$(Q|R)\\
Per induzione su |t|:\\

\textbf{Caso Base |t| = 0}

Allora t = $\varepsilon \in$\tr{P|R} $\Rightarrow$ $\varepsilon \in$\tr{Q|R}
\\

\textbf{Caso Induttivo |t| = n + 1}

t ha una prima interazione seguita poi dalla traccia t', quindi P|R $ \overset{\alpha_{1}}\rightarrow $ X' $\overset{t'}\rightarrow$ ovvero viene applicata una transizione secondo la regola del parallelo arrivando in un certo processo X'; ci sono perciò tre possibilità:

\begin{itemize}
	\item P|R ha effettuato una transizione usando la regola PAR-L:\\
	
	$\dfrac{P \overset{\alpha_{1}}\rightarrow P'}{P|R \overset{\alpha_{1}}\rightarrow P'|R\overset{t'}\rightarrow}$ \textit{PAR-L} \\
	
	t = $\alpha_{1}.t'$ con |t'| = n, per ipotesi induttiva t'$\in$\tr{P'|R}, \\
	si applica il lemma:\\
	\tr{P'} $\subseteq$ \tr{$\displaystyle\sum_{}^{} \{Q'|Q \overset{\alpha_{1}}\rightarrow Q'\}$} e quindi t' $\in$ \tr{P'|R} $\Rightarrow$ t'$\in$\tr{$\displaystyle\sum_{Q \overset{\alpha_{1}}\rightarrow Q'}^{} (Q'|R) $}
	
	Di conseguenza t$\in\alpha_{1}$.\tr{$\displaystyle\sum_{Q\overset{\alpha_{1}}\rightarrow Q'}^{} (Q'|R) $} = $\bigcup_{Q \overset{\alpha_{1}}\rightarrow Q'}\tr{\alpha_{1}.(Q'|R) $} e quindi, dato che il processo Q|R attraverso una transizione $\alpha_{1}$ può arrivare al processo Q'|R, si dimostra che se t$\in$(P|R) $\Rightarrow t\in$(Q|R).
	
	\item P|R ha effettuato una transizione usando la regola PAR-R:\\
	
	$\dfrac{R \overset{\alpha_{1}}\rightarrow R'}{P|R \overset{\alpha_{1}}\rightarrow P|R'\overset{t'}\rightarrow}$ \textit{PAR-R} \\
	
		t = $\alpha_{1}.t'$ con |t'| = n, per ipotesi induttiva t'$\in$\tr{P|R'},\\
		si applica il lemma:\\
		t'$\in$\tr{P|R'} $\Rightarrow$ t'$\in$\tr{Q|R'} e quindi t$\in$\tr{$\alpha_{1}$.(P|R')}$\Rightarrow$ t$\in$\tr{$\alpha_{1}$.(Q|R')}.
	 \\
		Dato che il processo Q|R attraverso una transizione $\alpha_{1}$ può arrivare al processo Q|R', si dimostra che se t$\in$(P|R) $\Rightarrow t\in$(Q|R).
	\\
	\item P|R ha effettuato una transizione usando la regola PAR-$\tau$:\\
	
	$\dfrac{P \overset{\alpha_{1}}\rightarrow P' \hspace{1cm} R \overset{\Overline[0.1pt]{\alpha_{1}}}\rightarrow R'}{P|R \overset{\tau_{1}}\rightarrow P'|R'\overset{t'}\rightarrow}$ \textit{PAR-$\tau$}
	
	
	t = $\tau_{1}.t'$ con |t'| = n, per ipotesi induttiva t'$\in$\tr{P'|R'}, \\
	si applica il lemma:\\
	\tr{P'} $\subseteq$ \tr{$\displaystyle\sum_{}^{} \{Q'|Q\overset{\alpha_{1}}\rightarrow Q'\}$} e quindi t'$\in$\tr{P'|R'} $\Rightarrow$ t'$\in$\tr{$\displaystyle\sum_{Q \overset{\alpha_{1}}\rightarrow Q'}^{} (Q'|R') $}
	
	Di conseguenza t$\in\tau_{1}$.\tr{$\displaystyle\sum_{Q \overset{\alpha_{1}}\rightarrow Q'}^{} (Q'|R') $} = $\bigcup_{Q \overset{\alpha_{1}}\rightarrow Q'}\tr{\tau_{1}.(Q'|R') $} e quindi, dato che il processo Q|R attraverso una transizione $\tau_{1}$ effettua la sincronizzazione tra Q e R per arrivare al processo Q'|R', si dimostra che se t$\in$(P|R) $\Rightarrow t\in$(Q|R).
	\\
\end{itemize}

$(\supseteq)$ 
\\

Sia t $\in$\tr{Q|R}  $\Rightarrow$  t $\in$(P|R)\\
Per induzione su |t|:\\

\textbf{Caso Base |t| = 0}

Allora t = $\varepsilon \in$\tr{Q|R} $\Rightarrow$ $\varepsilon \in$\tr{P|R}\\

\textbf{Caso Induttivo |t| = n + 1}

t ha una prima interazione seguita poi dalla traccia t', quindi Q|R $ \overset{\alpha_{1}}\rightarrow $ X' $\overset{t'}\rightarrow$ ovvero viene applicata una transizione secondo la regola del parallelo arrivando in un processo stato X'; ci sono perciò tre possibilità:

\begin{itemize}
	\item Q|R ha effettuato una transizione usando la regola PAR-L:\\
	
	$\dfrac{Q \overset{\alpha_{1}}\rightarrow Q'}{Q|R \overset{\alpha_{1}}\rightarrow Q'|R\overset{t'}\rightarrow}$ \textit{PAR-L} \\
	
	t = $\alpha_{1}.t'$ con |t'| = n, per ipotesi induttiva t'$\in$\tr{Q'|R}, \\
	si applica il lemma:\\
	\tr{Q'} $\subseteq$ \tr{$\displaystyle\sum_{}^{} \{P'|P \overset{\alpha_{1}}\rightarrow P'\}$} e quindi t' $\in$ \tr{Q'|R} $\Rightarrow$ t' $\in$ \tr{$\displaystyle\sum_{P \overset{\alpha_{1}}\rightarrow P'}^{} (P'|R) $}
	
	Di conseguenza t $\in$ $\alpha_{1}$.\tr{$\displaystyle\sum_{P \overset{\alpha_{1}}\rightarrow P'}^{} (P'|R) $} = $\bigcup_{P \overset{\alpha_{1}}\rightarrow Q'}\tr{\alpha_{1}.(P'|R) $} e quindi, dato che il processo P|R attraverso una transizione $\alpha_{1}$ può arrivare al processo P'|R, si dimostra che se t$\in$(Q|R) $\Rightarrow t \in$(P|R).
	
	\item P|R ha effettuato una transizione usando la regola PAR-R:\\
	
	$\dfrac{R \overset{\alpha_{1}}\rightarrow R'}{Q|R \overset{\alpha_{1}}\rightarrow Q|R'\overset{t'}\rightarrow}$ \textit{PAR-R} \\
	
	t = $\alpha_{1}.t'$ con |t'| = n, per ipotesi induttiva t' $\in \tr{Q|R'}$,\\
	si applica il lemma:\\
	t'$\in$\tr{Q|R'} $\Rightarrow$ t'$\in$\tr{P|R'} e quindi t$\in$\tr{$\alpha_{1}$.(Q|R')}$\Rightarrow$ t$\in$\tr{$\alpha_{1}$.(P|R')}.
	\\
	Dato che il processo P|R attraverso una transizione $\alpha_{1}$ può arrivare al processo P|R', si dimostra che se t$\in$(Q|R) $\Rightarrow t \in$(P|R).
	\\
	\item P|R ha effettuato una transizione usando la regola PAR-$\tau$:\\
	
	$\dfrac{Q \overset{\alpha_{1}}\rightarrow Q' \hspace{1cm} R \overset{\Overline[0.1pt]{\alpha_{1}}}\rightarrow R'}{Q|R \overset{\tau_{1}}\rightarrow Q'|R'\overset{t'}\rightarrow}$ \textit{PAR-$\tau$}
	
	
	t = $\tau_{1}.t'$ con |t'| = n, per ipotesi induttiva t' $\in \tr{Q'|R'} \Rightarrow t' \in \tr{P'|R'} $, \\
	si applica il lemma:\\
	\tr{Q'} $\subseteq$ \tr{$\displaystyle\sum_{}^{} \{P'|P \overset{\alpha_{1}}\rightarrow P'\}$} e quindi t' $\in$ \tr{Q'|R'} $\Rightarrow$ t' $\in$ \tr{$\displaystyle\sum_{P \overset{\alpha_{1}}\rightarrow P'}^{} (P'|R') $}
	
	Di conseguenza t $\in$ $\tau_{1}$.\tr{$\displaystyle\sum_{P \overset{\alpha_{1}}\rightarrow P'}^{} (P'|R') $} = $\bigcup_{P \overset{\alpha_{1}}\rightarrow P'}\tr{\tau_{1}.(P'|R') $} e quindi, dato che il processo P|R attraverso una transizione $\tau_{1}$ effettua la sincronizzazione tra P e R per arrivare al processo P'|R', si dimostra che se t$\in$(Q|R) $\Rightarrow t \in$(P|R).
	
\end{itemize}

Quindi con \tr{P} = \tr{Q}, P|R $\sim_{t}$ Q|R come si voleva dimostrare.

\paragraph{Contesto restrizione  C[ ] = \textbackslash L} \mbox{}

Il caso del contesto restrizione L sul processo P ha la seguente uguaglianza:\\ \tr{P\textbackslash L} = \tr{P}\textbackslash\{t=...$\alpha_{x}$...|$\alpha_{x}\in$L\} cioè le traccie che stanno in \tr{P} non ci sono nell'insieme di restrizione definito precedentemente. Se questo è vero, dato che per ipotesi \tr{P} = \tr{Q} e quindi \tr{Q\textbackslash L} = \tr{Q}\textbackslash \{t=...$\alpha_{x}$...|$\alpha_{x}\in$L\}, allora \tr{P\textbackslash L} = \tr{Q\textbackslash L}. 

Perciò si deve dimostrare che il contesto restrizione L sul processo P è uguale a \\
\tr{P}\textbackslash \{t=...$\alpha_{x}$...|$\alpha_{x}\in$L\}. Dimostrato questo ne consegue la veridicità di P\textbackslash L $\sim_{t}$ Q\textbackslash L.\\

$(\subseteq)$ \\

Sia t $\in$\tr{P\textbackslash L}  $\Rightarrow$  t $\in$\tr{P}\textbackslash\{t=...$\alpha_{x}$...|$\alpha_{x}\in$L\}\\
Per induzione su |t|:
\\

\textbf{Caso Base |t| = 0}

Allora t = $\varepsilon \in$\tr{P}\textbackslash\{t=...$\alpha_{x}$...|$\alpha_{x}\in$L\}
\\

\textbf{Caso Induttivo |t| = n + 1}

t ha una prima interazione seguita poi dalla traccia t', quindi P\textbackslash L $ \overset{\alpha_{1}}\rightarrow $ X' $\overset{t'}\rightarrow$ ovvero viene applicata una transizione secondo la regola della restrizione arrivando in un certo processo X', quindi:
	
	$\dfrac{P \overset{\alpha_{1}}\rightarrow P'}{P\textbackslash L \overset{\alpha_{1}}\rightarrow P'\textbackslash L\overset{t'}\rightarrow}$ \textit{RES} se $\alpha_{1}, \Overline[1pt]{\alpha_{1}} \not\in$ L\\
	
	t = $\alpha_{1}.t'$ con |t'| = n, per ipotesi induttiva t' $\in$ \tr{P'\textbackslash L}, quindi 
	$\alpha_{1}.t' \in \alpha_{1}$.\tr{P'\textbackslash L}$\subseteq$ \tr{P\textbackslash L}. Dato che $\alpha_{1}, \Overline[1pt]{\alpha_{1}} \not\in$ L quindi t = $\alpha$.t' $\not \in$ \{ t=...$\alpha_{x}$...|$\alpha_{x}\in$L\}  allora \\
	t $\in$\tr{P}\textbackslash \{t =...$\alpha_{x}$...|$\alpha_{x}\in$L\}.\\

$(\supseteq)$\\

Sia  t $\in$\tr{P}\textbackslash\{t=...$\alpha_{x}$...|$\alpha_{x}\in$L\} $\Rightarrow$ t $\in$\tr{P\textbackslash L}.   \\
Per induzione su |t|:
\\

\textbf{Caso Base |t| = 0}

Allora t = $\varepsilon \in$\tr{P\textbackslash L}
\\

\textbf{Caso Induttivo |t| = n + 1}

t ha una prima interazione seguita poi dalla traccia t', cioè t = $\alpha_{1}$.t'. Si ha quindi una transizione P $\overset{\alpha_{1}}\rightarrow$ P', ciò è permesso dalla regola della restrizione, quindi:

$\dfrac{P \overset{\alpha_{1}}\rightarrow P'}{P\textbackslash L \overset{\alpha_{1}}\rightarrow P'\textbackslash L\overset{t'}\rightarrow}$ \textit{RES} se $\alpha_{1}, \Overline[1pt]{\alpha_{1}} \not\in$ \{t=...$\alpha_{x}$...|$\alpha_{x}\in$L\}\\

Questo dimostra che P$\textbackslash$L sa fare l'interazione $\alpha_{1}$, perciò per ipotesi induttiva t'$\in$P'$\textbackslash$L allora $\alpha_{1}$.t' $\in$\tr{P$\textbackslash$L}.
\\

Quindi dato che \tr{P\textbackslash L} = \tr{P}\textbackslash \{t=...$\alpha_{x}$...|$\alpha_{x}\in$L\} con \tr{P} = \tr{Q}  allora si è dimostrato che  P\textbackslash L $\sim_{t}$ Q\textbackslash L.

\paragraph{Contesto relabelling  C[ ] = $\mathbf{[f]}$ } \mbox{}

Nel caso del contesto relabelling sul processo P si ha che:	\\	
Data la funzione $f: Act \rightarrow Act$, le traccie di P$\mathbf{[f]}$ sono:\\
f($\epsilon$) = $\epsilon$\\
f($\alpha.t$) =  f($\alpha$).f(t)\\
Quindi voglio dimostrare che \tr{P$\mathbf{[f]}$} = \{f(t)| t $\in$ \tr{P}\}. Se questo è vero, dato che \tr{P} = \tr{Q}  si può sostituire \tr{P} con \tr{Q} scrivendo \{f(t)| t $\in$ \tr{Q}\} e grazie alla uguaglianza scritta precedentemente, allora \tr{P$\mathbf{[f]}$} = \tr{Q$\mathbf{[f]}$}. \\

$(\subseteq)$ \\

Sia t$\in$ \tr{P$\mathbf{[f]}$} $\Rightarrow$ t $\in$ \{f(t)| t $\in$ \tr{P}\}. \\
Per induzione su |t|:
\\

\textbf{Caso Base |t| = 0}

Allora t = $\varepsilon \in$ \{f($\varepsilon$)| $\varepsilon$ $\in$ \tr{P}\}\\


\textbf{Caso Induttivo |t| = n + 1}

t ha una prima interazione seguita poi dalla traccia t', quindi P$\mathbf{[f]}  \overset{\alpha_{1}}\rightarrow $ X' $\overset{t'}\rightarrow$ ovvero viene applicata una transizione secondo la regola del relabelling arrivando in un certo processo X', quindi:

$\dfrac{P \overset{\alpha_{1}}\rightarrow P'}{P \mathbf{[f]} \overset{f(\alpha_{1})}\rightarrow P'\mathbf{[f]}\overset{f(t')}\rightarrow}$ \textit{REL} 

t = $f(\alpha_{1}).f(t)'$ con |t'| = n, per ipotesi induttiva si ha che f(t') $\in$ \{f(t')| t' $\in$ \tr{P'}\}, allora f($\alpha$).f(t') $\in$ \{f(t)| t $\in$ \tr{P}\}.\\

$(\supseteq)$ \\

Sia t$\in$\{f(t)| t $\in$ \tr{P}\}  $\Rightarrow$ t$\in$\tr{P$\mathbf{[f]}$}. \\
Per induzione su |t|:\\

\textbf{Caso Base |t| = 0}

Allora t = $\varepsilon \in$\tr{P$\mathbf{[f]}$}\\

\textbf{Caso Induttivo |t| = n + 1}

t ha una prima interazione seguita poi dalla traccia t', cioè t = $\alpha_{1}$.t'. Si ha quindi una transizione P $\overset{\alpha_{1}}\rightarrow$ P', ciò è permesso dalla regola del relabelling, quindi:

$\dfrac{P \overset{\alpha_{1}}\rightarrow P'}{P \mathbf{[f]} \overset{f(\alpha_{1})}\rightarrow P'\mathbf{[f]}\overset{f(t')}\rightarrow}$ \textit{REL} 

Questo dimostra che P$\mathbf{[f]}$ sa fare l'interazione $\alpha_{1}$, perciò per ipotesi induttiva t'$\in$P'$\mathbf{[f]}$ allora $\alpha_{1}$.t' $\in$\tr{P$\mathbf{[f]}$}.

Quindi dato che \tr{P$\mathbf{[f]}$} = \{f(t)| t $\in$ \tr{P}\} con \tr{P} = \tr{Q} allora si è dimostrato che  P$\mathbf{[f]}\sim_{t}$ Q$\mathbf{[f]}$.\\

\subsubsection{Conclusione}
Si è dimostrato con i vari casi della dimostrazione precedente, che per ogni possibile contesto che può essere usato, la trace equivalence risulta essere è una congruenza per il CCS. 
\pagebreak