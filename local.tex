%%%%%%%%%%%%%%
%  COSTANTI  %
%%%%%%%%%%%%%%

% In questa prima parte vanno definite le 'costanti' utilizzate soltanto da questo documento.
% Devono iniziare con una lettera maiuscola per distinguersi dalle funzioni.

\newcommand{\DocLGC}{Linguaggi per il global computing}

\newcommand{\DocTitle}{Esercizi B e D + Barbershop}

%\newcommand{\DocSite}{www.provincia.treviso.it}

\newcommand{\DocRedazione}{Federico Perin}

\newcommand{\DocMatr}{2029215}

\newcommand{\DocDate}{Ottobre 2021}

\newcommand{\DocDescription}{Descrizione del documento}

\newcommand{\zero}{\mathbf{0}}

\newcommand{\tr}[1]{Tr(#1)}

\newcommand{\sip}[1]{$Size_{p}(#1)$}

\newcommand{\sis}[1]{$Size_{s}(#1)$}

\newcommand{\out}[1]{\overline{#1}}

\newcommand{\upLimP}[1]{\mathcal{Size}_{P}(#1)}

\newcommand{\upLimS}[1]{\mathcal{Size}_{S}(#1)}

\newcommand\Overline[2][1pt]{%
	\begin{tikzpicture}[baseline=(a.base)]
	\node[inner xsep=0pt,inner ysep=1.5pt] (a) {$#2$};
	\draw[line width= #1] (a.north west) -- (a.north east);
	\end{tikzpicture}
}